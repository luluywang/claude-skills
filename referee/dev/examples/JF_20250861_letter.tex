\documentclass[11pt]{article}

% Essential packages
\usepackage[T1]{fontenc}
\usepackage[utf8]{inputenc}
\usepackage[margin=1in]{geometry}

% Formatting
\usepackage{setspace}
\onehalfspacing
\raggedbottom
\setlength{\parindent}{0pt}
\setlength{\parskip}{0.5em}

% Page style
\usepackage{fancyhdr}
\pagestyle{fancy}
\fancyhead{}
\fancyfoot[C]{\thepage}
\renewcommand{\headrulewidth}{0pt}

\begin{document}

Dear Antoinette,

Thank you very much for sending me the paper ``The Credit Card Spending Channel of Monetary Policy: Micro Evidence from Account-level Data,'' MS 20250861, to referee.

This paper investigates how monetary policy affects consumer spending through credit card interest rates. Using account-level data from the Y-14M dataset, the authors employ a Regression Kink Design that leverages contractual APR ceilings, and complement this with a macro-level instrumental variables analysis using high-frequency monetary policy shocks. The headline finding is a large spending semi-elasticity of $-8.7$ with respect to APR changes for accounts at their interest rate ceiling.

I recommend a rejection. The RKD identifies effects only for a highly selected subsample---accounts at the APR ceiling comprise less than 0.5\% of observations and are held by constrained borrowers with median credit scores in the bottom decile. Moreover, the spending estimates capture substitution across payment methods rather than true consumption effects, and the LP-IV analysis fails the exclusion restriction by conflating the partial equilibrium credit card channel with broader general equilibrium effects.

As written, I believe the paper falls short of the high standards of the Journal of Finance.

I hope these comments are useful.

Sincerely,

Lulu Wang

\end{document}
