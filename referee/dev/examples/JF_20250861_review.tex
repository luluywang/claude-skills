\documentclass[11pt]{article}

% Essential packages
\usepackage[T1]{fontenc}
\usepackage[utf8]{inputenc}
\usepackage[margin=1in]{geometry}
\usepackage{amsmath}
\usepackage{amssymb}
\usepackage{hyperref}

% Formatting
\usepackage{setspace}
\onehalfspacing
\raggedbottom

% Page style
\usepackage{fancyhdr}
\pagestyle{fancy}
\fancyhead{}
\fancyfoot[C]{\thepage}
\renewcommand{\headrulewidth}{0pt}

\begin{document}
\title{\vspace*{-1in}Referee Report for ``The Credit Card Spending Channel of
Monetary Policy: Micro Evidence from Account-level Data''}
\author{MS 20250861}
\date{\date{}}

\maketitle
\global\long\def\argmax{\operatornamewithlimits{argmax}}%

\global\long\def\argmin{\operatornamewithlimits{argmin}}%

\global\long\def\E{\mathbb{E}}%

\global\long\def\P{\mathbb{P}}%

\global\long\def\R{\mathbb{R}}%

\global\long\def\Cov{\text{Cov}}%

\global\long\def\Var{\text{Var}}%

\global\long\def\d{\text{ d}}%

\global\long\def\lvert#1{\left.#1\right|}%


\global\long\def\rvert#1{\left|#1\right.}%


\section{Summary of the Paper}

This paper investigates the impact of monetary policy on consumer spending through the credit card interest rate channel. Using a large, account-level panel (Y-14M), the authors employ two main identification strategies. First, they use a Regression Kink Design (RKD) that leverages contractual APR ceilings. This strategy identifies a Local Average Treatment Effect (LATE) of APR changes on spending for accounts near their maximum APR. Second, they complement this with a macro-level local projection (LP) analysis, instrumenting FFR changes with high-frequency monetary policy shocks.

The headline finding from the RKD is a very large spending semi-elasticity of $-8.7$, and a modest semi-elasticity of balances of around $-3.7$. The LP-IV analysis finds that a one percentage point increase in the short term policy rate leads to around a $10$\% decline in credit card spending 2 months after the shock.

\section{Major Comments}

This paper studies how credit card interest rates affect consumption and borrowing with a novel RKD design. However, I think the highly selected nature of the sample, the inability to observe spending on other payment methods, a weighting issue with the RKD, and the failure of the exclusion restriction for the LP-IV design leave me unconvinced on how large the credit card channel of monetary policy transmission truly is.

\subsection{Selection into the RKD Sample}

The RKD identifies effects only for a highly selected subsample of accounts. As shown in Figure 5, Panel (b), accounts at their maximum APR comprise less than 0.5\% of observations for most of the sample, reaching only 2\% at the peak of the 2022--2023 tightening cycle. More problematically, Table 1 reveals that this RKD sample is dramatically unrepresentative: the median credit score decile is 3 versus 6 in the full sample, and utilization rates are roughly double. These are accounts with high margins (approximately 20.7 pp. versus 13.8 pp. in the full sample), held by constrained borrowers who already carry high balances.

This selection matters because Table 5 shows that high-credit-score accounts -- the typical cardholder --- exhibit small and statistically insignificant spending responses to interest rate changes. The RKD elasticity of $-8.7$ is therefore a LATE for a population that does not search (or cannot search) for alternative borrowing opportunities. Extrapolating from this group to understand the broader credit card channel or to calibrate macroeconomic models is problematic. 

It may be possible to extrapolate from this sample with the careful use of propensity scores. But even if the paper does this, I would need a convincing story for why some borrowers have higher margins than others.

\subsection{The Spending Results Cannot Be Interpreted as Consumption Effects}

The RKD identifies an own-price elasticity measuring how spending \textit{on a specific card} responds when that card's APR rises. This necessarily bundles direct effects with substitution \textit{across payment methods}. When an account holder's credit card becomes more expensive, they may shift spending to debit cards, other credit cards, or cash. The $-8.7$ elasticity provides no information about the destination of this spending. This is problematic because it means it is not possible to extrapolate from this elasticity to how much interest rate increases affect aggregate demand through this credit card channel.

\subsection{Endogenous Weighting in the RKD Estimates}

The RKD estimates are weighted by an account-month's share of spending (or balance) in the subsequent month. Specifically, Table 2's note states: \textquotedbl Estimates are weighted by an account-month's share of total spending that occurred in the subsequent month, winsorized at the 97.5th percentile.\textquotedbl

This creates an econometric problem. The weight $w_{l,t}$ equals spending at $t+1$, which is the same variable appearing as the dependent variable $\ln(S_{l,t+1})$. Since the weight is a deterministic function of the outcome, it embeds the regression residual and violates the orthogonality condition $\mathbb{E}[w \cdot X \cdot u] = 0$ necessary for valid weighted regression. The weight and error term become mechanically correlated, potentially biasing estimates.

The paper should either (1) clarify the econometric validity of this weighting scheme, or (2) provide robustness checks using predetermined weights (e.g., based on spending at time $t$ or earlier). This is essential to establish that the reported elasticities are not biased artifacts of the weighting choice.

\subsection{The Local Projection Design Is Problematic}

The LP-IV analysis instruments FFR changes with high-frequency monetary policy shocks (Jaroci\'nski and Karadi 2020). Even if this instrument is relevant, it fails the exclusion restriction. However, monetary policy shocks affect many dimensions of the economy beyond credit card interest rates. These include income expectations, employment prospects, asset prices, housing wealth, and other credit conditions. An FFR shock generates simultaneous changes in all of these factors, and the LP-IV coefficient thus conflates the direct credit card channel with all general equilibrium effects operating through these other channels.

In other words, the IV approach does not isolate the partial equilibrium credit card channel from macro effects. A policymaker wanting to understand how much monetary policy passes through specifically via credit cards cannot reliably read this off the LP-IV coefficients. Additional structure would be required to decompose these effects.

\section{Minor Issues}

In addition to the major points above, the paper requires clarification on several technical details:

\begin{itemize}

    \item Even if the data can only be used to study the residual demand elasticity of borrowing with respect to the interest rate, that might still be very interesting. A long standing question in the literature is why credit card interest rates are so high despite competition. Understanding the residual demand elasticity could shed light on this puzzle. Admittedly, the current estimate of the balance semi-elasticity of $-3.7$ implies a markup of almost 25 percentage points, which seems too high. One reason could be that the outside option is getting worse as the policy rate increases (which is the key source of variation used in the paper). 

    \item I think throughout the estimand is the percentage change in X for a 1 percentage point increase in the interest rate. This should be referred to as a semi-elasticity.
    
    \item The mathematical definition of $\tau$ displays $\lim_{d\to0^\pm} c(d)/d$ (ratio of levels) rather than $\lim_{d\to0^\pm} \partial c(d)/\partial d$ (derivatives). Unless outcomes are normalized to zero at the kink, these ratio limits do not represent slope changes and are inconsistent with standard RKD notation. The formula should either show derivatives or explicitly state that variables are demeaned at $d=0$.

    \item The text states that $\beta_1$ is ``the derivative of credit card spending with respect to the distance as the distance approaches zero from above.'' However, the piecewise regression structure implies $\beta_1$ is the slope for $d < 0$ (approaching from below); the derivative from above is $\beta_1 + \beta_2$. This reversal should be corrected.

    \item  The figure title for figure A-2 reads ``Income Distribution by Credit-Score Decile,'' but the note states it displays ``percentiles of interest-rate margin.'' These are different variables; the title and note must be aligned (most likely the note should say ``income'').

    
\end{itemize}

\end{document}
