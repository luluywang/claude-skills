%% LyX 2.4.2.1 created this file.  For more info, see https://www.lyx.org/.
%% Do not edit unless you really know what you are doing.
\documentclass[11pt,twoside,english,figuresright]{article}
\usepackage[T1]{fontenc}
\usepackage[latin9]{inputenc}
\pagestyle{empty}
\usepackage{babel}
\usepackage{amsmath}
\usepackage{amsthm}
\usepackage{geometry}
\geometry{verbose,tmargin=1in,bmargin=1in,lmargin=1in,rmargin=1in}
\usepackage{setspace}
\usepackage[authoryear]{natbib}
\onehalfspacing
\usepackage[bookmarks=true,bookmarksnumbered=false,bookmarksopen=false,
 breaklinks=false,pdfborder={0 0 1},backref=false,colorlinks=false]
 {hyperref}
\hypersetup{
 citecolor=blue}

\makeatletter
%%%%%%%%%%%%%%%%%%%%%%%%%%%%%% User specified LaTeX commands.
\usepackage{footnote}
\usepackage{url}
\newcounter{trow}
\newcommand{\firstrow}{\setcounter{trow}{1}\arabic{trow}.}
\newcommand{\nextrow}{\addtocounter{trow}{1}\arabic{trow}.}
\usepackage{comment}
\usepackage{mathptmx}
\usepackage{indentfirst}
\usepackage{rotating}
\usepackage{hyperref}
\let\oldFootnote\footnote
\newcommand\nextToken\relax
\renewcommand\footnote[1]{%
\oldFootnote{#1}\futurelet\nextToken\isFootnote}
\newcommand\isFootnote{%
\ifx\footnote\nextToken\textsuperscript{,}\fi}
\usepackage{fancyhdr}
\pagestyle{fancy}
\fancyhead{}
\fancyfoot{}
\fancyfoot[C]{\thepage}
\renewcommand{\headrulewidth}{0pt}
\usepackage{tikz}
\def\checkmark{\tikz\fill[scale=0.4](0,.35) -- (.25,0) -- (1,.7) -- (.25,.15) -- cycle;}
\usepackage{caption}
\captionsetup{labelfont=bf, margin=1cm, tableposition=top}

\usepackage{microtype}

\raggedbottom

% Stuff for submission

\usepackage{ifthen}

\makeatother

\begin{document}
\title{\vspace*{-1in}Referee Report for ``Risk-Based Borrowing Limits in
Credit Card Markets''}
\author{MS 25-00233}
\date{\date{}}

\maketitle
\global\long\def\argmax{\operatornamewithlimits{argmax}}%

\global\long\def\argmin{\operatornamewithlimits{argmin}}%

\global\long\def\E{\mathbb{E}}%

\global\long\def\P{\mathbb{P}}%

\global\long\def\R{\mathbb{R}}%

\global\long\def\Cov{\text{Cov}}%

\global\long\def\Var{\text{Var}}%

\global\long\def\d{\text{ d}}%

\global\long\def\lvert#1{\left.#1\right|}%
 

\global\long\def\rvert#1{\left|#1\right.}%


\section{Summary of the Paper}

The paper studies the effects of tailored credit limits and interest
rates in the credit card market. The author first introduces novel
statement-level panel data from the UK where she can observe borrower
characteristics, contract terms, borrowing, and default. This data
reveals that lenders offer borrowers vastly different credit limits,
but very similar prices. The author builds on this fact with a model
to explore the roles of credit limits and interest rates in screening,
as well as the effects of offering personalized interest rates in
this market. The key finding of the model is that allowing lenders
to customize prices leads to substantial dispersion in interest rates
as lenders charge substantially higher interest rates to low-income
borrowers. Rates rise both because such borrowers are less price elastic,
and also because they have worse observable credit risk.

\section{Major Comments}

The topic of how and why lenders in credit markets customize some
contract terms, and not others, is incredibly important. A fleshed
out empirical exercise that helps us understand the separate roles
of interest rates and credit limits, and when is it ok to model one
but not the other, would significantly advance the literature.
\begin{itemize}
\item The state of the art modeling of credit card markets is in Nelson
(2025, ECMA forthcoming). In that model, borrowers make dynamic discrete
choices over whether or not to borrow, and from what lender. However,
in that model there is no intensive margin decision of how much to
borrow and thus no role for credit limits to shape borrowing and default
behavior. The author's paper innovates by modeling both the credit
limit and the interest rate.
\item Other work by Arthur Taburet (``Screening Using a Menu of Contracts'',
JFE Accepted) models competition in multiple contract terms, but with
arguably less natural contract terms. Taburet shows that lenders can
offer menus of prices and maturities to screen borrowers. But a long
literature going back to Stiglitz and Weiss highlights the role of
controlling quantities. The author's paper innovates by offering a
path towards studying an important quantity: credit limits.
\end{itemize}
However, the paper suffers from three important problems. First, the
paper is not very transparent on what the incremental contribution
is and how that contribution follows from the quantitative estimation
of the model. Second, although the paper sets out a goal of comparing
the roles of interest rate and credit limit customization, the model
makes several assumptions that seem to bake in the result that lenders
have weak incentives to customize prices and strong incentives to
customize credit limits. Third, consumers' inability to switch between
banks after credit limits are revealed makes it difficult to evaluate
the role of heterogeneous lender information in this market.

\subsection{What is the incremental contribution?}

The introduction argues the paper makes several contributions but
many of the results are either not novel or almost tautological to
the model.
\begin{itemize}
\item Although the UK setting is stark in having very little price variation,
work in the US has also highlighted the relatively limited variation
in prices compared to credit limits. For example, in the summary statistics
to the original Agarwal et al CARD act papers (``Do Banks Pass Through
Credit Expansions to Consumers Who want to Borrow'') they highlight
credit limits that vary from \$2,000 \textendash{} \$7,000 as one
goes up the credit score distribution but interest rates that are
essentially flat. 
\item The author finds large shadow costs of customizing interest rates,
but this is essentially a baked in result that arises from the lack
of interest rate variation in the data.
\item The author finds that low income individuals have low price elasticities.
However, this is consistent with work in Nelson (2025) that consumers
vary both by their default risk and their demand elasticity, so that
limitations on lender pricing both reduces the scope for risk-based
pricing and price discrimination. Hastings (2017, ``Sales Force and
Competition in Financial Product Markets'') also find similar patterns
where low income individuals have low price elasticies.
\item The author finds that tailoring interest rates increases profits by
23\%, hurts price inelastic consumers, and helps price elastic consumers.
But this is a necessary result from any model in which lenders go
from charging uniform prices to differentiated prices.
\end{itemize}
One question that would be novel and could be answered by the paper
is the question ``what is more important: giving lenders flexibility
over credit limits or over interest rates?'' I do not seem to get
a clear answer to this question in the current draft. I want to see
a more clearly laid out research question whose answer depends more
on the quantitative results.

\subsection{To what extent is the result on interest rates vs credit limits baked
in?}

The model of consumer choice treats credit limits and interest rates
in different ways. This makes it hard to know if the results on comparing
interest rates versus credit limits are baked in the model or if they're
the result of some correlations in the data.

The model has two stages. First, consumers make an extensive margin
decision of whether to borrow on a credit card or to just transact
on the credit card to earn rewards. Second, consumers make an intensive
margin decision of how much to borrow on the credit card. In the first
stage, consumers pick a card based on the advertised interest rate,
but in the second stage consumers make an optimal borrowing decision
based on the adjusted interest rate and the credit limit.

In this world, it makes sense to use interest rates to tailor to demand
because it is the only instrument that can target heterogeneous price-sensitivities
at the extensive margin adoption stage. Credit limits then have the
responsibility of screening for default and controlling for loss given
default at the intensive margin stage. How would this change if interest
rates didn't matter at the extensive margin stage, or if credit limits
also mattered at the shopping stage? I understand data limits may
make it difficult to study the effects of expected credit limits on
extensive margin choices, but it seems important to put both variables
on equal footing in the demand model if the point of the model is
to compare the two types of screening.

I see two potential ways to address this issue:
\begin{enumerate}
\item I would be ok with a model in which consumers were exogenously matched
to lenders at the ``shopping stage'', and then consumers make decisions
of what to do in response to price and credit limit customization
at the intensive margin stage. This would let the demand types in
the intensive margin stage shape the results of whether lenders want
to customize with credit limits or interest rates, rather than a strong
assumption that credit limits don't matter at the extensive margin
stage.
\item I would also be ok with a demand model in which consumers care about
credit limits and interest rates at the shopping stage. Is there potentially
a way to extrapolate from the PPI IV at the shopping stage to infer
the price-sensitivity of borrowing at the intensive margin? Is there
potentially a way to extrapolate from the credit limit sensitivity
at the intensive margin stage to a credit limit sensitivity at the
shopping stage?
\end{enumerate}
Given the importance of the topic I'm comfortable with strong assumptions
to make the estimation work. My main concern right now is that the
model is set up in a way that bakes in the result on the roles of
interest rates and credit limits.

\subsection{The Role of Lender Information}

Although the model innovates on lenders' information structures, the
model's assumptions on the intensive margin borrowing decisions seems
to understate the importance of heterogeneous information in this
market. 

In the model, lenders do not compete in credit limits. That is, when
a lender is tailoring a credit limit he only needs to consider how
it changes the utility of borrowing on the credit card relative to
the outside option and not relative to another lender's card. 

Since borrowers do not substitute to other lenders after credit limits
are set, lenders do not need to worry about a winner's curse as in
Shaffer (1998, ``The Winner's Curse in Banking''). These winner's
curse considerations are are another potential reason why lenders
with more precise signal structures are much more likely to serve
high risk consumers. Lenders with imprecise signals cannot serve these
markets because their low-risk consumers will get better offers from
the lenders with precise signals. For related work on this issue,
see Cherry (2025, ``Regulating Credit: The Impact of Price Regulations
and Lender Technologies on Financial Inclusion'', Working Paper).

My main recommendation is to de-emphasize the role of heterogeneous
information sets. It is an absolutely fascinating topic but it seems
tangential to the roles of credit limits versus interest rates.

\section{Minor Comments}
\begin{itemize}
\item I really like the use of intensive margin behavior to think about
risk types. It does feel intuitive that the people who need the full
credit limit are more likely to be desperate to borrow. They may be
in a state where they are then more likely to default.
\item The paper is way too long. The introduction takes 3 pages to get to
a result. The model does not appear until page 21. On page 43 I'm
being asked to evaluate the plausbility of a cost shifter.
\item What do I need to believe for the outside good of revolving on a credit
card to be transacting, instead of using a debit card or transitioning
to some other lender's credit card? What do I lose with this assumption?
\item I don't know what it means to say that ``interest rates are risk-based
to maximize interest revenues, and credit limits are risk-based to
cover downside default risk''. Both tools are part of the profit
maximization problem, and the interesting question is whether the
two terms are valued differently by consumers with heterogeneous default
risks.
\item Instead of describing modeling simplifications as ``profitable deviations'',
just be transparent on the costs and benefits. Let the reader make
the judgment on whether it's a good simplification.
\item A lot of credit card regulation is about price regulation and not
credit limit regulation. How much price regulation can lenders undo
by adjusting on credit limits?
\end{itemize}

\end{document}
